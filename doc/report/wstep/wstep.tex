\section{Wstęp}
\subsection{Wprowadzenie}
Kierunek badań odtwarzania sygnału rzadkiego (ang. \textbf{CS} – \textit{Compressed Sensing}) jest stosunkowo nową i bardzo ciekawym kierunkiem badań w~dziedzinie przetwarzania sygnałów cyfrowych (ang. \textbf{DSP} – \textit{Digital Signal Processing}). W klasycznym podejściu, aby dokładnie odtworzyć sygnał ciągły, próbkuje się go przynajmniej z częstotliwością Nyquista. Dla sygnałów szybkozmiennych jest to jednak bardzo kosztowne, gdyż wymaga to znacznego skomplikowania czujników (wyposażając je w mechanizm kompresji danych) lub wymusza wykorzystanie kanału transmisyjnego o dużej przepustowości. Często również wykonanie samego pomiaru jest drogie i powolne (jak w przypadku rezonansu magnetycznego) lub niebezpieczne (promienie rentgenowskie). Wyniki badań z dziedziny próbkowania rzadkiego dowodzą jednak, że często udaje się dokładnie odtworzyć sygnał ze znacznie mniejszej liczby pomiarów, wykorzystując jego nadmiarowość i rzadką reprezentację w odpowiednio dobranej przestrzeni \cite{IntroductionCS}. Ta sama idea wykorzystywana jest w algorytmach kompresji danych: rzadka reprezentacja zdjęcia w bazie falkowej jest metodą zmniejszenia wielkości obrazu w standardzie JPEG2000 \cite{JPEG2000}. Algorytmy CS są zasadniczo różne - próbują one znaleźć taki sposób pomiaru, który pobiera konieczne informacje o sygnale już w skompresowanej formie. Proces odtwarzania sygnału możemy więc podzielić na dwie podstawowe czynności - rzadkie próbkowanie i rekonstrukcję. Drugi krok, wykorzystujący numeryczne metody optymalizacji, jest wysoce nietrywialny. 

Poniższy raport składa się z sześciu rozdziałów. Po wstępie następuje rozdział poświęcony złożoności czasowej i pamięciowej postawionego problemu. Rozdział trzeci przedstawia koncepcję jego rozwiązania, a czwarty - procedury testowe. Rozdziały 5 i 6 opisują odpowiednio wyniki przeprowadzonych eksperymentów oraz podsumowanie. Szczegółowy opis zadania znajduje się w załączniku.
\subsection{Cele i założenia projektu}
Głównym celem projektu była implementacja wybranego algorytmu odtwarzania sygnału rzadkiego, wykorzystując do obliczeń kartę graficzną. Zadaniem dodatkowym była symulacja jednopikselowej kamery, wykorzystując kamerę firmy Jai, dostępną w sali laboratoryjnej. Projekt miał umożliwić znaczne przyspieszenie istniejących rozwiązań, opartych głównie o skrypty środowiska Matlab/Simulink, a także umożliwić odtwarzanie obrazów o większej skali (docelowo obraz w pełnej rozdzielczości kamery, czyli 2560 x 2048 pikseli). Z uwagi na przyjęty sposób generowania pomiarów oraz przede wszystkim - konieczność przechowywania macierzy pomiarowej w pamięci, okazało się to kompletnie nierealne.
\subsection{Zarys proponowanego rozwiązania}
Projekt rozpoczęto od rozległych badań literaturowych. Oprócz zrozumienia zagadnienia, próbowano rozeznać się w dostępnych metodach, a w szczególności – ich potencjalnej wydajności mplementuje algorytm TQVC[Link]. Jest on atrakcyjny ze względu na stosunkową łatwość w implementacji, jak i wysoką skuteczność. Funkcją celu, minimalizowaną w tej metodzie jest  totalna wariacja (ang. \textbf{TV} – \textit{Total Varation}). Jej istota polega na założeniu, że \textit{gradient} zdjęcia jest rzadki.